% libbibneat.tex -- Overview of the bibneat C++ Library API
\documentclass[11pt]{article}
\usepackage[a4paper,margin=2.5cm]{geometry}
\usepackage{hyperref}
\usepackage{enumitem}
\usepackage{listings}
\usepackage{tcolorbox}
\usepackage{parskip}
\setlength{\parindent}{0pt}

\usepackage{xcolor}
\definecolor{codegreen}{rgb}{0,0.6,0}
\definecolor{codegray}{rgb}{0.5,0.5,0.5}
\definecolor{codepurple}{rgb}{0.58,0,0.82}
\definecolor{backcolour}{rgb}{0.95,0.95,0.92}

\lstset{
    backgroundcolor=\color{backcolour},   
    commentstyle=\color{codegreen},
    keywordstyle=\color{magenta},
    numberstyle=\tiny\color{codegray},
    stringstyle=\color{codepurple},
    basicstyle=\ttfamily\footnotesize,,
    columns=fullflexible,
}

\title{\texttt{libbibneat} \\ \large A C++ Library for BibTeX Manipulation}
\author{Lorenzo Peri}
\date{\today}

\begin{document}
\maketitle

\begin{abstract}
\texttt{libbibneat} is a C++ shared library for parsing, transforming, and writing BibTeX files. It powers the \texttt{bibneat} and \texttt{bibgrab} tools, and is designed to be a flexible foundation for your own bibliography-processing projects.
\end{abstract}

\section{Philosophy and Design}
The core idea of \texttt{libbibneat} is to provide a modular, composable API for working with BibTeX data. The main container is a \texttt{BibDB}, which holds many \texttt{BibEntry} objects (each representing a single BibTeX entry) via a vector of \texttt{std::unique\_ptr}. All transformations and parsing are performed by objects that operate on a \texttt{BibDB}.

\section{Core Data Structures}
\subsection{\texttt{BibEntry} and \texttt{BibDB}}
\begin{itemize}
  \item \texttt{BibEntry} (\texttt{lib/bibneat/database/bibfile.hpp}): Represents a single BibTeX entry. Stores the entry type, key, and all fields as strings. Provides methods to add attributes and determine BibTeX type.
  \item \texttt{BibDB} (\texttt{lib/bibneat/database/bibfile.hpp}): A container for many \texttt{BibEntry} objects (and special entries). Provides methods to add, clear, and manage entries.
\end{itemize}

\section{API Components}
All major API classes take a \texttt{std::shared\_ptr<BibDB>} in their constructor and operate on the database in-place.

\subsection{Parsing}
\textbf{Header:} \texttt{lib/bibneat/database/parser.hpp}

\begin{itemize}
  \item \texttt{Parser}: Parses BibTeX files (\texttt{Parser::parse(std::string filename)}) or strings \\
(\texttt{Parser::parseString(std::string bibTex)}) and adds entries to a \texttt{BibDB}.
\end{itemize}

\subsection{API Calls}
\textbf{Header:} \texttt{lib/bibneat/recipes/apicalls.hpp}

\begin{itemize}
  \item \texttt{ApiCaller}: Connects to arXiv and doi.org APIs.
    \begin{itemize}
      \item \texttt{ApiCaller::checkArXiv(bool replaceWithArxiv, bool replaceWithDOI)} -- Checks arXiv entries, can replace with published version.
      \item \texttt{ApiCaller::checkDOI(bool replaceWithDOI)} -- Checks DOI entries, can replace with published version.
    \end{itemize}
\end{itemize}

\subsection{Field Filtering}
\textbf{Header:} \texttt{lib/bibneat/recipes/fieldfilter.hpp}

\begin{itemize}
  \item \texttt{FieldFilter}: Removes fields unknown to BibTeX. Use \texttt{FieldFilter::keepBibTex()} to keep only standard fields.
\end{itemize}

\subsection{Unicode and Normalization}
\textbf{Header:} \texttt{lib/bibneat/uni/fieldnormalization.hpp}

\begin{itemize}
  \item \texttt{FieldNormalizer}: Handles Unicode normalization and LaTeX encoding.
    \begin{itemize}
      \item \texttt{FieldNormalizer::NFCNormalize()} -- Canonical normalization.
      \item \texttt{FieldNormalizer::NFKCNormalize()} -- Compatibility normalization.
      \item \texttt{FieldNormalizer::uni2latex()} -- Converts Unicode to LaTeX encoding (NFD decomposition, combining character mapping, special character mapping, NFC composition).
      \item \texttt{FieldNormalizer::addUTF8Preamble()} -- Adds a UTF-8 encoding preamble to the BibDB.
    \end{itemize}
\end{itemize}

\subsection{Printing}
\textbf{Header:} \texttt{lib/bibneat/database/printer.hpp}

\begin{itemize}
  \item \texttt{Printer}: Writes the BibDB to a .bib file (\texttt{Printer::toBibFile(std::string filename, bool overwrite=true)}).
\end{itemize}

\section{Example Usage}
\begin{lstlisting}[language=C++]
#include <memory>
#include "lib/bibneat/database/bibfile.hpp"
#include "lib/bibneat/database/parser.hpp"
#include "lib/bibneat/recipes/apicalls.hpp"
#include "lib/bibneat/recipes/fieldfilter.hpp"
#include "lib/bibneat/uni/fieldnormalization.hpp"
#include "lib/bibneat/database/printer.hpp"

int main() {
    // Create a new BibDB (the main bibliography container)
    auto db = std::make_shared<BibDB>();
    // Parse a .bib file and populate the BibDB
    Parser parser(db);
    parser.parse("myrefs.bib");
    // Set up API caller for arXiv/DOI lookups (timeouts in seconds)
    long connTimeout=15L;
    long totTimeout=20L;
    ApiCaller api(db, connTimeout, totTimeout);
    // Check arXiv entries and replace with published version if available
    bool replaceWithArxiv=true;
    bool replaceWithDOI=true;
    api.checkArXiv(replaceWithArxiv, replaceWithDOI);
    // Remove non-standard BibTeX fields
    FieldFilter filter(db);
    filter.keepBibTex();
    // Normalize Unicode and convert to LaTeX encoding
    FieldNormalizer normalizer(db);
    normalizer.NFCNormalize();
    normalizer.uni2latex();
    // Add UTF-8 encoding preamble
    normalizer.addUTF8Preamble();
    // Write the cleaned bibliography to a file
    Printer printer(db);
    printer.toBibFile("cleaned.bib");
}
\end{lstlisting}

\section{Extending libbibneat}
The API is designed to be modular and extensible. You can add your own transformation classes that operate on a \texttt{BibDB}, or use the provided utilities for string and Unicode handling (see \texttt{lib/bibneat/utils/}).

\section{License}
\texttt{libbibneat} is open source and distributed under the GLP3 License. Have fun!

\end{document}
